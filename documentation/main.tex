%%%%%%%%%%%%%%%%%%%%%%%%%%%%%%%%%%%%%%%%%%%%%%%%%%%%%%%%%%%%%%%%%%%%%%
% How to use writeLaTeX: 
%
% You edit the source code here on the left, and the preview on the
% right shows you the result within a few seconds.
%
% Bookmark this page and share the URL with your co-authors. They can
% edit at the same time!
%
% You can upload figures, bibliographies, custom classes and
% styles using the files menu.
%
%%%%%%%%%%%%%%%%%%%%%%%%%%%%%%%%%%%%%%%%%%%%%%%%%%%%%%%%%%%%%%%%%%%%%%

\documentclass[12pt]{article}

\usepackage{sbc-template}

\usepackage{graphicx,url}

%\usepackage[brazil]{babel}   
\usepackage[utf8]{inputenc}  

     
\sloppy

\title{Travelling Salesperson Problem:\\ Different Approaches and Performance in Exponential Time}

\author{Juan M. Braga F.\inst{1}}


\address{Departamento de Ciência da Computação -- Universidade Federal de Minas Gerais
  (UFMG)\\
  Av. Pres. Antônio Carlos, 6627 -- Pampulha -- Belo Horizonte -- MG -- Brazil
  \email{juanbraga@ufmg.br}
}

\begin{document} 

\maketitle

\begin{abstract}
  This paper documents the practical aspects of implementing solutions
  for the Travelling Salesperson Problem, a well known difficult problem that requires
  exponential time for the optimal solution. Three algorithms were implemented:
  an exact solution through the branch-and-bound technique, and two other approximations,
  the twice-around-the-tree and Christofides algorithms.
\end{abstract}
     
\begin{resumo} 
  Este artigo documenta os aspectos práticos da implementação de soluções
  para o Problema do Caixeiro Viajante, um problema bem conhecido e difícil que requer
  tempo exponencial para encontrar a solução ótima. Foram implementados três algoritmos:
  uma solução exata através da técnica de branch-and-bound, e duas outras aproximações,
  os algoritmos twice-around-the-tree e Christofides.
\end{resumo}


\section{Introducing the Travelling Salesperson Problem}

The Travelling Salesperson Problem (TSP) is aptly described by [BRILLIANT] 

\begin{quote}
  "A salesperson needs to visit a set of cities to sell their goods. They know how 
  many cities they need to go to and the distances between each city. In what order 
  should the salesperson visit each city exactly once so that they minimize their 
  travel time and so that they end their journey in their city of origin?"
\end{quote}

The TSP is a well known \textbf{intractable problem}, which in the context of 
algorithms means that the time and/or space required to solve the problem grows 
exponentially with the size of the input, making it impractical to solve even for 
relatively small inputs.

This project documents the development of three solutions for the given problem: 
branch-and-bound, twice-around-the-tree, and Christofides algorithms. It focuses 
on evaluating the real-world challenges associated with them, such as planning and 
optimizing them, making informed decisions on data structures and libraries, and 
analyzing the use of resources.

\section{First Page} \label{sec:firstpage}

The first page must display the paper title, the name and address of the
authors, the abstract in English and ``resumo'' in Portuguese (``resumos'' are
required only for papers written in Portuguese). The title must be centered
over the whole page, in 16 point boldface font and with 12 points of space
before itself. Author names must be centered in 12 point font, bold, all of
them disposed in the same line, separated by commas and with 12 points of
space after the title. Addresses must be centered in 12 point font, also with
12 points of space after the authors' names. E-mail addresses should be
written using font Courier New, 10 point nominal size, with 6 points of space
before and 6 points of space after.

The abstract and ``resumo'' (if is the case) must be in 12 point Times font,
indented 0.8cm on both sides. The word \textbf{Abstract} and \textbf{Resumo},
should be written in boldface and must precede the text.

\section{CD-ROMs and Printed Proceedings}

In some conferences, the papers are published on CD-ROM while only the
abstract is published in the printed Proceedings. In this case, authors are
invited to prepare two final versions of the paper. One, complete, to be
published on the CD and the other, containing only the first page, with
abstract and ``resumo'' (for papers in Portuguese).

\section{Sections and Paragraphs}

Section titles must be in boldface, 13pt, flush left. There should be an extra
12 pt of space before each title. Section numbering is optional. The first
paragraph of each section should not be indented, while the first lines of
subsequent paragraphs should be indented by 1.27 cm.

\subsection{Subsections}

The subsection titles must be in boldface, 12pt, flush left.

\section{Figures and Captions}\label{sec:figs}


Figure and table captions should be centered if less than one line
(Figure~\ref{fig:exampleFig1}), otherwise justified and indented by 0.8cm on
both margins, as shown in Figure~\ref{fig:exampleFig2}. The caption font must
be Helvetica, 10 point, boldface, with 6 points of space before and after each
caption.

\begin{figure}[ht]
\centering
\includegraphics[width=.5\textwidth]{fig1.jpg}
\caption{A typical figure}
\label{fig:exampleFig1}
\end{figure}

\begin{figure}[ht]
\centering
\includegraphics[width=.3\textwidth]{fig2.jpg}
\caption{This figure is an example of a figure caption taking more than one
  line and justified considering margins mentioned in Section~\ref{sec:figs}.}
\label{fig:exampleFig2}
\end{figure}

In tables, try to avoid the use of colored or shaded backgrounds, and avoid
thick, doubled, or unnecessary framing lines. When reporting empirical data,
do not use more decimal digits than warranted by their precision and
reproducibility. Table caption must be placed before the table (see Table 1)
and the font used must also be Helvetica, 10 point, boldface, with 6 points of
space before and after each caption.

\begin{table}[ht]
\centering
\caption{Variables to be considered on the evaluation of interaction
  techniques}
\label{tab:exTable1}
\includegraphics[width=.7\textwidth]{table.jpg}
\end{table}

\section{Images}

All images and illustrations should be in black-and-white, or gray tones,
excepting for the papers that will be electronically available (on CD-ROMs,
internet, etc.). The image resolution on paper should be about 600 dpi for
black-and-white images, and 150-300 dpi for grayscale images.  Do not include
images with excessive resolution, as they may take hours to print, without any
visible difference in the result. 

\section{References}

Bibliographic references must be unambiguous and uniform.  We recommend giving
the author names references in brackets, e.g. \cite{knuth:84},
\cite{boulic:91}, and \cite{smith:99}.

The references must be listed using 12 point font size, with 6 points of space
before each reference. The first line of each reference should not be
indented, while the subsequent should be indented by 0.5 cm.

\bibliographystyle{sbc}
\bibliography{sbc-template}

\end{document}
